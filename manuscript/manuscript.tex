\RequirePackage{etoolbox}
\newtoggle{draft}
\settoggle{draft}{true}

\iftoggle{draft}{%
	\documentclass[12pt,draftcls,onecolumn]{IEEEtran}
	\def\Figratio{0.8\linewidth}
	\usepackage{amsthm}
}{%
	\documentclass[journal,twoside,web]{ieeecolor}
	\def\Figratio{1\linewidth}
	\usepackage{journalformat}
}

\usepackage{mystyle}


\begin{document}

\title{%
	Data-Efficient Active Weighting Algorithm
	for Composite Adaptive Control Systems
}
\author{%
	Seong-hun~Kim,
	and~Youdan~Kim,~\IEEEmembership{Senior Member,~IEEE}
	\thanks{%
		This work was supported
		by a National Research Foundation of Korea (NRF)
		grant funded by the Korean government (MSIT)
		(No. 2019R1A2C208394612).
	}
	\thanks{%
		S-h. Kim, and Y. Kim are with the
		Institute of Advanced Aerospace Technology,
		Department of Aerospace Engineering,
		Seoul National University,
		Seoul,
		08826,
		Republic of Korea
	(e-mail: ydkim@snu.ac.kr).
	}
}

\maketitle


%==============================================================================
%		Abstract
%==============================================================================
\begin{abstract}
	Abstract... %TODO
\end{abstract}


%==============================================================================
%		Keywords
%==============================================================================
\begin{IEEEkeywords}
	Keywords... %TODO
\end{IEEEkeywords}

%==============================================================================
%		Introduction
%==============================================================================
\section{Introduction}

\IEEEPARstart{M}{odel-based} an be achieved even if residuals exist.
\cite{anderson_exponential_1977}


%==============================================================================
%		Acknowledgment
%==============================================================================
\section*{Acknowledgment}

This work was supported
by a National Research Foundation of Korea (NRF)
grant funded by the Korean government (MSIT)
(No. 2019R1A2C208394612).


\bibliographystyle{IEEEtran}
\bibliography{mybib}

% % Generated by IEEEtran.bst, version: 1.14 (2015/08/26)
% \begin{thebibliography}{10}
% \providecommand{\url}[1]{#1}
% \providecommand{\newblock}{\relax}
% \providecommand{\bibinfo}[2]{#2}
% \providecommand{\BIBentrySTDinterwordspacing}{\spaceskip=0pt\relax}
% \providecommand{\BIBentryALTinterwordstretchfactor}{4}
% \providecommand{\BIBdecl}{\relax}
% \BIBdecl

% \bibitem{astrom_adaptive_1987}
% K.~J. Astrom, ``Adaptive feedback control,'' \emph{Proc. IEEE}, vol.~75, no.~2,
%   pp. 185--217, 1987.

% \bibitem{slotine_composite_1989}
% J.-J.~E. Slotine and W.~Li, ``Composite adaptive
% control of robot manipulators,'' \emph{Automatica},
%   vol.~25, no.~4, pp. 509--519, 1989.

% \bibitem{slotine_applied_1991}
% J.-J. Slotine and W.~Li, \emph{Applied Nonlinear
% Control}. {Prentice Hall}, {Englewood Cliffs, NJ}, 1991.

% \bibitem{anderson_exponential_1977}
% B.~Anderson, ``Exponential stability of linear equations arising in adaptive
%   identification,'' \emph{IEEE Trans. Autom. Control}, vol.~22, no.~1, pp.
%   83--88, 1977.

% \bibitem{nguyen_optimal_2008}
% N.~Nguyen, K.~Krishnakumar, and J.~Boskovic, ``An
% optimal control modification to model-reference adaptive control for fast
% adaptation,'' in \emph{{{AIAA Guidance}},
% {{Navigation}} and {{Control Conference}}}, {Honolulu, HI}, Aug. 2008.

% \bibitem{boyd_necessary_1986}
% S.~Boyd and S.~Sastry, ``Necessary and sufficient conditions for parameter
%   convergence in adaptive control,'' \emph{Automatica}, vol.~22, no.~6, pp.
%   629--639, 1986.

% \bibitem{yucelen_derivative-free_2011}
% T.~Yucelen and A.~J. Calise, ``Derivative-free model
% reference adaptive control,'' \emph{J. Guid. Control
% Dyn.}, vol.~34, no.~4, pp. 933--950, 2011.

% \bibitem{duarte_combined_1989}
% M.~A. Duarte and K.~S. Narendra, ``Combined direct and indirect approach to
%   adaptive control,'' \emph{IEEE Trans. Autom. Control}, vol.~34, no.~10, pp.
%   1071--1075, 1989.

% \bibitem{lavretsky_combined/composite_2009}
% E.~Lavretsky, ``Combined/composite model reference adaptive control,''
%   \emph{IEEE Trans. Autom. Control}, vol.~54, no.~11, pp. 2692--2697, 2009.

% \bibitem{lavretsky_predictor-based_2010}
% E.~Lavretsky, R.~Gadient, and I.~M. Gregory,
% ``Predictor-based model reference adaptive
% control,'' \emph{J. Guid. Control Dyn.}, vol.~33,
%   no.~4, pp. 1195--1201, 2010.

% \bibitem{song_adaptive_2017}
% Y.~Song, K.~Zhao, and M.~Krstic, ``Adaptive control with exponential regulation
%   in the absence of persistent excitation,'' \emph{IEEE Trans. Autom. Control},
%   vol.~62, no.~5, pp. 2589--2596, 2017.

% \bibitem{kim_mismatch-observer_2017}
% S.-h. Kim and Y.~Kim, ``Mismatch-observer based model reference adaptive
%   control for transient performance improvement of aircraft,'' in \emph{{{AIAA
%   Guidance}}, {{Navigation}}, and {{Control Conference}}}.\hskip 1em plus 0.5em
%   minus 0.4em\relax {Grapevine, TX}: {American Institute of Aeronautics and
%   Astronautics}, Jan. 2017.

% \bibitem{gibson_adaptive_2013}
% T.~E. Gibson, A.~M. Annaswamy, and E.~Lavretsky, ``On adaptive control with
%   closed-loop reference models: Transients, oscillations, and peaking,''
%   \emph{IEEE Access}, vol.~1, pp. 703--717, 2013.

% \bibitem{yucelen_low-frequency_2013}
% T.~Yucelen and W.~M. Haddad, ``Low-frequency learning and fast adaptation in
%   model reference adaptive control,'' \emph{IEEE Trans. Autom. Control},
%   vol.~58, no.~4, pp. 1080--1085, 2013.

% \bibitem{yucelen_improving_2014}
% T.~Yucelen, G.~D.~L. Torre, and E.~N. Johnson,
% ``Improving transient performance of adaptive control
% architectures using frequency-limited system error dynamics,''
%   \emph{Int. J. Control}, vol.~87, no.~11, pp.
%   2383--2397, 2014.

% \bibitem{kamalapurkar_model-based_2016}
% R.~Kamalapurkar, P.~Walters, and W.~E. Dixon,
% ``Model-based reinforcement learning for approximate
% optimal regulation,'' \emph{Automatica}, vol.~64,
%   pp. 94--104, 2016.

% \bibitem{chowdhary_concurrent_2013}
% G.~Chowdhary, T.~Yucelen, M.~M{\"u}hlegg, and E.~N. Johnson,
% ``{Concurrent learning adaptive control of
% linear systems with exponentially convergent bounds},''
% \emph{International Journal of Adaptive
% Control \& Signal Processing}, vol.~27, no.~4, pp. 280--301, 2013.

% \bibitem{chowdhary_concurrent_2010}
% G.~Chowdhary and E.~Johnson, ``Concurrent learning for convergence in adaptive
%   control without persistency of excitation,'' in \emph{49th {{IEEE
%   Conference}} on {{Decision}} and {{Control}} ({{CDC}})}, {Atlanta, GA}, Dec.
%   2010.

% \bibitem{chowdhary_exponential_2014}
% G.~Chowdhary, M.~M{\"u}hlegg, and E.~Johnson, ``Exponential parameter and
%   tracking error convergence guarantees for adaptive controllers without
%   persistency of excitation,'' \emph{Int. J. Control}, vol.~87, no.~8, pp.
%   1583--1603, 2014.

% \bibitem{chowdhary_singular_2011}
% G.~Chowdhary and E.~Johnson, ``A singular value maximizing data recording
%   algorithm for concurrent learning,'' in \emph{Proceedings of the 2011
%   {{American Control Conference}}}, {San Francisco, CA}, Jun. 2011.

% \bibitem{pan_composite_2018}
% Y.~Pan and H.~Yu, ``Composite learning robot control with guaranteed parameter
%   convergence,'' \emph{Automatica}, vol.~89, pp. 398--406, 2018.

% \bibitem{cho_composite_2018}
% N.~Cho, H.~Shin, Y.~Kim, and A.~Tsourdos, ``Composite model reference adaptive
%   control with parameter convergence under finite excitation,'' \emph{IEEE
%   Trans. Autom. Control}, vol.~63, no.~3, pp. 811--818, 2018.

% \bibitem{nadler_finite_2008}
% B.~Nadler, ``Finite sample approximation results for
% principal component analysis: {{A}} matrix perturbation approach,''
% 	\emph{Ann. Statist.}, vol.~36, no.~6, pp.
% 	2791--2817, Dec. 2008.

% \bibitem{ipsen_refined_2009}
% I.~C.~F. Ipsen and B.~Nadler, ``Refined perturbation
% bounds for eigenvalues of hermitian and non-hermitian matrices,''
%   \emph{SIAM J. Matrix Anal. Appl.}, vol.~31, no.~1,
%   pp. 40--53, 2009.

% \bibitem{lavretsky_robust_2013}
% E.~Lavretsky and K.~Wise, \emph{Robust and {{Adaptive
% Control}}: {{With Aerospace Applications}}}, ser. Advanced {{Textbooks}} in
% 	{{Control}} and {{Signal Processing}}. {Springer-Verlag}, {London}, 2013.

% \bibitem{paternain_online_2017}
% S.~Paternain and A.~Ribeiro, ``Online learning of feasible strategies in
%   unknown environments,'' \emph{IEEE Trans. Autom. Control}, vol.~62, no.~6,
%   pp. 2807--2822, 2017.

% \bibitem{bauschke_convex_2011}
% H.~H. Bauschke and P.~L. Combettes, \emph{Convex Analysis and Monotone Operator
%   Theory in {{Hilbert}} Spaces}, ser. {{CMS}} Books in Mathematics.\hskip 1em
%   plus 0.5em minus 0.4em\relax {New York}: {Springer}, 2011.

% \bibitem{khalil_nonlinear_2002}
% H.~K. Khalil, \emph{Nonlinear Systems}, 3rd~ed.
% {Prentice Hall}, {Upper Saddle River, NJ}, 2002.

% \bibitem{belsley_regression_1980}
% D.~A. Belsley, E.~Kuh, and R.~E. Welsch, \emph{Regression diagnostics:
% 	identifying influential data and sources of collinearity}, ser. Wiley series
% 	in probability and mathematical statistics. {Wiley}, {NY}, 1980.

% \bibitem{elzebda_development_1989}
% J.~M. Elzebda, A.~H. Nayfeh, and D.~T. Mook,
% ``Development of an analytical model of wing rock for
% slender delta wings,'' \emph{Journal of Aircraft},
%   vol.~26, no.~8, pp. 737--743, 1989.

% \end{thebibliography}

\begin{IEEEbiography}
	[{\includegraphics[width=1in,height=1.25in,clip,keepaspectratio]{SKim.jpg}}]
	{Seong-hun Kim}
	received the B.S. degree
	in Mechanical and Aerospace Engineering
	from Seoul National University,
	the Republic of Korea,
	in February 2015,
	where he is currently pursuing the Ph.D. degree.
	His current research interests include
	robust adaptive control based on online parameter learning,
	machine learning applications for flight control,
	and data-driven optimal control and trajectory optimization for UAVs.
\end{IEEEbiography}

\begin{IEEEbiography}
	[{\includegraphics[width=1in,height=1.25in,clip,keepaspectratio]{YKim.pdf}}]
	{Youdan Kim} (M'94--SM'16)
	received the B.S. and M.S. degrees
	in Aeronautical Engineering
	from Seoul National University,
	the Republic of Korea,
	in 1983 and 1985, respectively,
	and a Ph.D. in Aerospace Engineering
	from Texas A\&M University in 1990.
	He joined the Faculty of Seoul National University in 1992,
	where he is currently a professor at the Department of Aerospace Engineering.
	His current research interests include
	nonlinear flight control, reconfigurable control, path planning,
	and guidance techniques for aerospace applications.
\end{IEEEbiography}

\end{document}
