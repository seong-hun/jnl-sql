\documentclass[12pt,draftcls,onecolumn]{IEEEtran}

%-------------------------------------------------------------------------------
%		Packages
%-------------------------------------------------------------------------------
\usepackage{cite}
\usepackage{amsmath,amssymb,amsfonts}
\usepackage{textcomp}
\usepackage{algorithm}
\usepackage{algpseudocode}

\usepackage{mathtools}
\usepackage{physics}
\usepackage{bbm}
\usepackage{booktabs,multirow}
\usepackage{siunitx}
\sisetup{table-figures-exponent=1, table-sign-exponent=true}
\sisetup{group-separator={,}}

%-------------------------------------------------------------------------------
%		Graphicsx
%-------------------------------------------------------------------------------
\usepackage{graphicx}
\usepackage[caption=false, font=footnotesize]{subfig}
\graphicspath{{src/}}

%------------------------------------------------------------------------------
%		AMS Theorems
%------------------------------------------------------------------------------
\usepackage{amsthm}
\newtheorem{thm}{Theorem}
\newtheorem{lem}{Lemma}
\newtheorem{cor}{Corollary}
\newtheorem{prop}{Proposition}

\newtheorem{defi}{Definition}
\newtheorem{prob}{Problem}
\newtheorem{exam}{Example}
\newtheorem{alg}{Algorithm}
\newtheorem{ass}{Assumption}

\newtheorem{remark}{Remark}
\newtheorem{note}{Note}

%------------------------------------------------------------------------------
%		Macros
%------------------------------------------------------------------------------
\DeclarePairedDelimiterX{\inner}[2]{\langle}{\rangle}{#1, #2}

%------------------------------------------------------------------------------
%		Document
%------------------------------------------------------------------------------
\begin{document}

\title{Research Note}
\author{Seong-hun~Kim}
\date{\today}

\maketitle

%==============================================================================
\section{Introduction}
%==============================================================================

Consider a class of linear systems as follows.
\begin{equation}
  \label{eq:system}
  \dot{x} = A x + B u,\; x(0) = x_0
\end{equation}
Suppose that
the system $(A, B)$ is controllable.
The performance index
for the Linear-Quadratic-Regulator (LQR) problem
is given by
\begin{equation}
  J(x_0; u(\cdot)) = \int_0^\infty \frac{1}{2} \pqty{x^T Q x + u^T R u} \dd{t},
\end{equation}
where
$(\sqrt{Q}, A)$ is detectable,
and the matrix $R$ is positive definite.

The purpose of this study is
to find the optimal gain $K^\ast$ for the LQR problem,
using an arbitrary exploration data
which is a set of a tuple $(x, u, \dot{x})$.
The exploration data can either be obtained
from a single trajectory controlled by a human expert,
or even collected from literally any data point $(x, u, \dot{x})$
that satisfies the relation in~\eqref{eq:system}.

An algorithm is \textit{off-policy}
if it learns the optimal policy
from this kind of exploration data.
Several off-policy algorithms have been proposed
to obtain $K^\ast$ for linear systems
\cite{%
yu_jiang_robust_2014,
jiang_global_2015,
jae_young_lee_integral_2015,
vamvoudakis_q-learning_2017
}.
Most of these algorithms are based
on a Kleinman algorithm~\cite{kleinman_iterative_1968},
which requires an initial stabilizing policy
as follows.

\begin{thm}[Kleinman~\cite{kleinman_iterative_1968}]
  Let $K_0$ be
  any stabilizing feedback gain matrix,
  and let $P_k$ be
  the symmetric positive definite solution
  of the Lyapunov equation given by
  \begin{equation}
    P_k (A - B K_k) + (A - B K_k)^T P_k + Q + K_k^T R K_k = 0,
  \end{equation}
  where $K_k$ is defined recursively by
  \begin{equation}
    K_k = R^{-1} B^T P_{k-1}.
  \end{equation}
  Then, the following properties hold.
  \begin{enumerate}
    \item $A - B K_k$ is Hurwitz,
    \item $P^\ast \le P_{k+1} \le P_k$,
    \item $\lim_{k\to\infty} = K^\ast$,
      and $\lim_{k\to\infty} = P^\ast$.
  \end{enumerate}
\end{thm}

Online learning methods
which adapt the gain
using the Hamilton-Jacobi-Bellman (HJB) error have been proposed
to relax the requirement
for the initial stabilizing policy~\cite{vamvoudakis_online_2010}.
However,
these methods only guarantee
the UUB for gain errors,
and the ultimate bound is too large
to claim that
the gain converges to the optimal gain.

%==============================================================================
\section{Problem Formulation}
%==============================================================================

Let us consider the following linear system.
\begin{equation}
	\dot{x}(t) = A x(t) + B u(t), \; x(0) = x_0,
\end{equation}
where
$x \in \mathbb{R}^n$ is the state vector,
and
$u \in \mathbb{R}^m$ is the input vector.
The matrices $A$ and $B$
with proper dimensions
are unknowns. 

The Linear Quadratic Regulator (LQR) finds
a controller $u$
that minimizes the following cost functional.
\begin{equation}
	J(x(0); u(\cdot))
	= \int_0^\infty
	\frac{1}{2} \pqty{{x(\tau)}^T Q x(\tau) + {u(\tau)}^T R u(\tau)} \dd{\tau},
\end{equation}
where
$Q \in \mathbb{S}_+^n$
and
$R \in \mathbb{S}_{++}^m$ are fixed matrices.
Assume that
$\pqty{A, B}$ is controllable,
and $\pqty{\sqrt{Q}, A}$ is detectable.
It is well known that
the optimal controller has
a linear feedback form as
$u^\ast(t) = - K x(t)$
with the optimal gain $K = R^{-1} B^T P$,
where
$P \in \mathbb{S}_{++}^n$
is the solution
to the following algebraic Riccati equation.
\begin{equation}
  \label{eq:are}
	P A + A^T P + Q - P B R^{-1} B^T P = 0.
\end{equation}

For the optimal gain $K$,
consider the following
control dynamics-augmented system.
\begin{align}
	\label{eq:augmented_system}
	\dot{x} & = A x + B u,
	\\
	\dot{u} & = F u + F K x - K \pqty{Ax + Bu},
\end{align}
where
the matrix $F \in \mathbb{R}^m$ is Hurwitz.

\begin{lem}
	For the augmented system~\eqref{eq:augmented_system},
	the value function is given by
	\begin{equation}
		\mathcal{Q}(x, u)
		= \frac{1}{2} x^T P x + \frac{1}{2} {(u + Kx)}^T G (u + Kx)
	\end{equation}
\end{lem}

%==============================
\section{Structured Q-Learning}
%==============================

Given a gain matrix $K$,
consider the following
control-dynamics-augmented system.
\begin{align}
  \dot{x} & = A x + B u, \\
  \dot{u} & = F (u + K x) - K \pqty{A x + B u},
\end{align}
where $F$ is Hurwitz.
This system is a virtual system
that has no effect
to the original system in~\eqref{eq:system}.
Only the learning algorithm exploits this system.

Let $\mathcal{Q}(x, u; K)$ denote
the value function (Q-function)
for the augmented system
corresponding to the gain $K$,
which satisfies the following Bellman equation.
\begin{equation}
  \label{eq:Bellman_equation}
  {\grad_x \mathcal{Q}(x, u)}^T \dot{x}
  + {\grad_u \mathcal{Q}(x, u)}^T \dot{u}
  + \frac{1}{2} \pqty{x^T Q x + u^T R u} 
  = 0.
\end{equation}
If 
$\mathcal{Q}_k(x, u) \coloneqq \mathcal{Q}(x, u; K_k)$,
the Q-function for the gain $K_k$,
is structured as
\begin{equation}
  \mathcal{Q}_k(x, u)
  =
  \frac{1}{2}
  \bmqty{x^T & u^T}
  \bmqty{P_k^\prime & W_k^T \\ W_k & G_k}
  \bmqty{x \\ u}
  \eqqcolon
  \frac{1}{2} z^T H_k z,
\end{equation}
where $z = [x^T, u^T]^T$,
the Bellman equation
in~\eqref{eq:Bellman_equation}
can be written as follows.
\begin{equation}
  \label{eq:evaluation}
  z^T H_k \bmqty{\dot{x} \\ F (u + K_k x) - K_k \dot{x}}
  + \frac{1}{2} \pqty{x^T Q x + u^T R u} 
  = 0.
\end{equation}
The variables
$x$, $u$, $z$, $\dot{x}$, $F$, $K_k$, $Q$, and $R$
are known,
the solution $H_k$ can be found if it exists.
Algorithm~\ref{alg:structured_q_learning}
recursively solves for $H_k$
and updates $K_k$
which is hopefully guaranteed
to converge to the optimal gain $K^\ast$
without any knowledge of the system
and the initial stabilizing policy.

\begin{algorithm}
  \caption{Structured Q-Learning}
  \label{alg:structured_q_learning}
  \begin{algorithmic}
    \State Initialize the gain matrix $K_0$.
    \State $k = 0$
    \Repeat
    \State Find the symmetric matrix $H_k$
    using~\eqref{eq:evaluation}.
    \Comment{Policy Evaluation}
    \State Policy update: $K_{k+1} = {(H_k)}_{22}^{-1} {(H_k)}_{21}$.
    \Comment{Policy Improvement}
    \State $k \gets k + 1$
    \Until{forever}
  \end{algorithmic}
\end{algorithm}

Figure~\ref{fig:comparison:a}
compares
the existing method
(Z. P. Jiang~\cite{jiang_global_2015})
based on Kleinman algorithm
and the proposed method
for the initial stable policy,
while
Fig.~\ref{fig:comparison:b}
shows the same comparison
for the initial unstable policy.
The data is collected
from a single trajectory
with arbitrarily oscillating control inputs.
It can be seen that
the two methods have same convergence
when the initial policy is stable.
However,
for the unstable initial policy,
the only proposed method
converges to the optimal.

\begin{figure}[!t]
  \label{fig:comparison}
	\centering
  \subfloat[Stable initial policy\label{fig:comparison:a}]{%
	\includegraphics[width=0.48\linewidth]{figure_1.pdf}}
	\hfill
  \subfloat[Unstable initial policy\label{fig:comparison:b}]{%
	\includegraphics[width=0.48\linewidth]{figure_2.pdf}}
	\caption{State, input, and parameter convergence histories.}
\end{figure}

%=============================
\section{Convergence Analysis}
%=============================

The control gain is updated
using the previous Q-function parameters as
\begin{equation}
  \label{eq:update_law}
  K_{k+1} = {(H_k)}_{22}^{-1} {(H_k)}_{21}
\end{equation}
The evaluation step~\eqref{eq:evaluation} calculates
the Q-function,
which can be represented as
the following matrix equation.
\begin{equation}
	\label{eq:ith_HJB_equation}
	H_k {A_k} + {A_k}^T H_k + Q^\prime = 0,
\end{equation}
where
\begin{equation}
	H_k = \bmqty{P_k^{\prime} & W_k^T \\ W_k & G_k}
  \eqqcolon
  \bmqty{(H_k)_{11} & (H_k)_{21}^T \\ (H_k)_{21} & (H_k)_{22}},
  \quad
  Q^\prime \coloneqq \bmqty{Q & 0 \\ 0 & R},
\end{equation}
and
\begin{equation}
  A_k
  \coloneqq
  \bmqty{A & B \\ F K_k - K_k A & F - K_k B}.
\end{equation}

To simplify the analysis,
the matrix equation~\eqref{eq:ith_HJB_equation} is transformed
using the following decompositions:
\begin{equation}
  \label{eq:Aiprime_def}
  A_k
  =
  \bmqty{I & 0 \\ - K_k & I}
  \bmqty{A - B K_k & B \\ 0 & F}
  \bmqty{I & 0 \\ K_k & I},
\end{equation}
and
\begin{equation}
	\label{eq:Hi_def}
	H_k
	=
  \bmqty{I & W_k^T {G_k}^{-1} \\ 0 & I} 
	\bmqty{P_k & 0 \\ 0 & G_k}
  \bmqty{I & 0 \\ {G_k}^{-1} W_k & I}
	=
	\bmqty{I & K_{k+1}^T \\ 0 & I} 
	\bmqty{P_k & 0 \\ 0 & G_k}
	\bmqty{I & 0 \\ K_{k+1} & I},
\end{equation}
where $P_k \coloneqq P_k^\prime - W_k^T G_k^{-1} W_k$,
from Schur's decomposition.
By substituting $A_k^\prime$ in~\eqref{eq:Aiprime_def}
and $H_k^\prime$ in~\eqref{eq:Hi_def}
into~\eqref{eq:ith_HJB_equation},
we get
\begin{equation}
  \label{eq:rearranged_iteration}
  \tilde{H}_k
	\bmqty{A - B K_k & B \\ 0 & F}
	+
	\bmqty{\pqty{A - B K_k}^T & 0 \\ B^T & F^T}
  \tilde{H}_k
  +
	\bmqty{I & -K_{k}^T \\ 0 & I}
	Q^\prime
	\bmqty{I & 0 \\ -K_k & I}
  = 0
\end{equation}
where $\tilde{K}_i = K_{k+1} - K_k$,
and
\begin{align}
  \label{eq:Htilde_def}
  \tilde{H}_k
  & =
  \bmqty{I & \tilde{K}_{k}^T \\ 0 & I}
  \bmqty{P_k & 0 \\ 0 & G_k}
  \bmqty{I & 0 \\ \tilde{K}_{k} & I}
  =
  \bmqty{%
    P_k + \tilde{K}_k^T G_k \tilde{K}_k & \tilde{K}_k^T G_k
    \\
    G_k \tilde{K}_k & G_k
  }
  \\
  & \eqqcolon
  \bmqty{%
    (\tilde{H}_k)_{11} & (\tilde{H}_k)_{21}^T
    \\
    (\tilde{H}_k)_{21} & (\tilde{H}_k)_{22}
  }.
\end{align}
Then,
the update law~\eqref{eq:update_law} can be rewritten as
\begin{equation}
  K_{k+1} = K_k + {(\tilde{H}_k)}_{22}^{-1} {(\tilde{H}_k)}_{21}.
\end{equation}

\begin{figure}[!t]
  \label{fig:comparison2}
  \centering
  \subfloat[Stable initial policy\label{fig:comparison2:a}]{%
  \includegraphics[width=0.48\linewidth]{figure_3.pdf}}
  \hfill
  \subfloat[Unstable initial policy\label{fig:comparison2:b}]{%
  \includegraphics[width=0.48\linewidth]{figure_4.pdf}}
  \caption{Parameter convergence histories.}
\end{figure}

Rewrite the equation in~\eqref{eq:rearranged_iteration}
using each block matrix of $\tilde{H}_k$ as
\begin{gather}
  \label{eq:block_1}
  (\tilde{H}_k)_{11} (A - B K_k)
  + (A - B K_k)^T (\tilde{H}_k)_{11}
  + Q + K_k^T R K_k = 0,
  \\
  \label{eq:block_2}
  (\tilde{H}_k)_{21} (A - B K_k) + B^T (\tilde{H}_k)_{11}
  + F^T (\tilde{H}_k)_{21} - R K_k = 0,
  \\
  \label{eq:block_3}
  (\tilde{H}_k)_{21} B + (\tilde{H}_k)_{22} F
  + F^T (\tilde{H}_k)_{22} + B^T (\tilde{H}_k)_{21}^T + R = 0.
\end{gather}
Multiply
$(\tilde{H}_k)_{21}^T (\tilde{H}_k)_{22}^{-1}$
on the left hand side of~\eqref{eq:block_2} and~\eqref{eq:block_3},
and
multiply
$(\tilde{H}_k)_{22}^{-1} (\tilde{H}_k)_{21}$
on the right hand side of~\eqref{eq:block_3}.
Then,
by substituting both equations into~\eqref{eq:block_1},
we have
\begin{equation}
  \label{eq:H11_k+1}
  \mathcal{P}(\tilde{H}_k) (A - B K_{k+1})
  + (A - B K_{k+1})^T \mathcal{P}(\tilde{H}_k) 
  + Q + K_{k+1}^T R K_{k+1} = 0,
\end{equation}
where
$\mathcal{P}(\tilde{H}_k)
\coloneqq
(\tilde{H}_k)_{11}
- (\tilde{H}_k)_{21}^T (\tilde{H}_k)_{22}^{-1} (\tilde{H}_k)_{21}
$.
It can be observed
that $\mathcal{P}(\tilde{H}_k) = (\tilde{H}_{k+1})_{11} = P_k$
from~\eqref{eq:Htilde_def} and~\eqref{eq:block_1}.

\begin{thm}
  Suppose that
  $(\sqrt{Q}, A)$ is observable, $Q \succeq 0$, and $R \succ 0$.
  Let $H_k$, $k=0,1,\dotsc$, be
  the solution to the Lyapunov equation~\eqref{eq:ith_HJB_equation}
  where $K_0$ is chosen such that the matrix $A - BK_0$ is Hurwitz.
  Then,
  $P_k = (H_k)_{11} - (H_k)_{21}^T (H_k)_{22}^{-1} (H_k)_{21}^T$
  satisfies
  \begin{enumerate}
    \item $\cdots \succeq P_k \succeq P_{k+1} \succeq \cdots \succ 0$
    \item $\lim_{k\to\infty} P_k = P^\ast$
      and $\lim_{k\to\infty} K_k = K^\ast$,
      where $P^\ast$ is the maximal solution of~\eqref{eq:are},
      and $K^\ast = R^{-1} B^T P^\ast$ is the stable optimal gain.
  \end{enumerate}
\end{thm}

\begin{proof}
  1) Let $K_k$ stabilize the system $(A, B)$.
  Since $F$ is Hurwitz,
  the augmented matrix
  \begin{equation}
    \tilde{A}_k \coloneqq \bmqty{A - B K_k & B \\ 0 & F}
  \end{equation}
  is also Hurwitz.
  From the fact that
  $X \succeq 0$ is equivalent to $S^T X S \succeq 0$
  for any non-singular matrix $S$,
  we have
  \begin{equation}
    \tilde{Q}_k
    \coloneqq
    \bmqty{I & -K_{k}^T \\ 0 & I}
    \bmqty{Q & 0 \\ 0 & R}
    \bmqty{I & 0 \\ -K_k & I}
    \succeq 0.
  \end{equation}
  Then,
  there exists a unique nonnegative solution $\tilde{H}_k$
  to~\eqref{eq:rearranged_iteration}
  \cite[Theorem 2.2]{snyders_nonnegative_1970}.
  Moreover,
  because the pair $(\tilde{Q}_k, \tilde{A}_k)$ is observable,
  the solution $\tilde{H}_k$ is positive definite.
  From the definition of $\tilde{H}_k$ in~\eqref{eq:Htilde_def},
  $(\tilde{H}_k)_{11}
  - (\tilde{H}_k)_{21}^T (\tilde{H}_k)_{22}^{-1} (\tilde{H}_k)_{21}
  = P_k
  \succ0$
  and
  $(\tilde{H}_k)_{22} = G_k \succ 0$.
  From~\eqref{eq:H11_k+1},
  $(\tilde{H}_{k+1})_{11}$ and $(\tilde{H}_{k})_{11}$ have
  the following relation.
  \begin{equation}
    \label{eq:H11_relation}
    (\tilde{H}_{k+1})_{11}
    =
    (\tilde{H}_k)_{11}
    - (\tilde{H}_k)_{21}^T (\tilde{H}_k)_{22}^{-1} (\tilde{H}_k)_{21}
    \succ 0,
  \end{equation}
  which implies that
  $(\tilde{H}_k)_{11} \succeq (\tilde{H}_{k+1})_{11} = P_k \succ 0$.
  Since
  $\tilde{H}_{k+1} \succ 0$,
  $\tilde{Q}_{k+1} \succeq 0$
  and $(\tilde{Q}_{k+1}, \tilde{A}_{k+1})$ is observable,
  from Lasalle's theorem,
  the matrix $\tilde{A}_{k+1}$ is Hurwitz,
  which implies that the gain $K_{k+1}$ is stabilizing.
  By induction, we can complete the proof.

  2) There exists $\lim_{k\to\infty} P_k = P_\infty$
  and also $\lim_{k\to\infty} (\tilde{H}_k)_{11} = (\tilde{H}_\infty)_{11}$
  from the monotonic convergence of positive operators%
  ~\cite{riesz_functional_1990}.
  By taking the limit $k\to\infty$ of~\eqref{eq:H11_relation},
  we have $(\tilde{H}_\infty)_{21} = 0$
  since $(\tilde{H}_\infty)_{22} \succ 0$,
  which implies
  $K_k = R^{-1} B^T (\tilde{H}_\infty)_{11}$
  from~\eqref{eq:block_2}.
  Then,
  the limit of \eqref{eq:block_1}
  turns out to be the algebraic Riccati equation
  \begin{equation}
    (\tilde{H}_\infty)_{11} A
    + A^T (\tilde{H}_\infty)_{11}
    + Q - (\tilde{H}_\infty)_{11} B R^{-1} B^T (\tilde{H}_\infty)_{11} = 0,
  \end{equation}
  where $A - B R^{-1} B (\tilde{H}_\infty)_{11}$ is Hurwitz,
  which possess the unique positive-definite solution $P^\ast$.
\end{proof}


\bibliographystyle{IEEEtran}
\bibliography{manuscript}

\end{document}
