\documentclass[12pt,draftcls,onecolumn]{IEEEtran}


%-------------------------------------------------------------------------------
%		Packages
%-------------------------------------------------------------------------------
\usepackage{cite}
\usepackage{amsmath,amssymb,amsfonts}
\usepackage{textcomp}
\usepackage{algorithmic}

\usepackage{mathtools}
\usepackage{physics}
\usepackage{bbm}
\usepackage{booktabs,multirow}
\usepackage{siunitx}
\sisetup{table-figures-exponent=1, table-sign-exponent=true}
\sisetup{group-separator={,}}

%-------------------------------------------------------------------------------
%		Graphicsx
%-------------------------------------------------------------------------------
\usepackage{graphicx}
\graphicspath{{src/}}

%------------------------------------------------------------------------------
%		AMS Theorems
%------------------------------------------------------------------------------
\usepackage{amsthm}
\newtheorem{thm}{Theorem}
\newtheorem{lem}{Lemma}
\newtheorem{cor}{Corollary}
\newtheorem{prop}{Proposition}

\newtheorem{defi}{Definition}
\newtheorem{prob}{Problem}
\newtheorem{exam}{Example}
\newtheorem{alg}{Algorithm}
\newtheorem{ass}{Assumption}

\newtheorem{remark}{Remark}
\newtheorem{note}{Note}

%------------------------------------------------------------------------------
%		Macros
%------------------------------------------------------------------------------
\DeclarePairedDelimiterX{\inner}[2]{\langle}{\rangle}{#1, #2}

%------------------------------------------------------------------------------
%		Document
%------------------------------------------------------------------------------
\begin{document}

\title{Note}
\author{Seong-hun~Kim}

\maketitle

%==============================================================================
\section{Introduction}
%==============================================================================

Consider the following nonlinear system.
\begin{equation}
	\label{eq:system}
	\dot{x} = f(x) + g(x) u, \quad x(0) = x_0
\end{equation}
where
$x \in \Omega \subset \mathbb{R}^n$ denotes the state vector,
and
$u \in \mathbb{R}^m$ denotes the control input vector.
The functions
$f \colon \Omega \to \mathbb{R}^n$
and
$g \colon \Omega \to \mathbb{R}^{n \times m}$
are assumed to be Lipschitz continuous on a set $\Omega$
which contains the origin as an interior point.

Let the performance index of the control be defined by
\begin{equation}
	\label{eq:performance_index}
	J(x, u) = \int_{0}^\infty r(\varphi(t; x, u), u(t)) \dd{t},
\end{equation}
where
$r \colon \Omega \times \mathbb{R}^m \to \mathbb{R}$ denotes
the local cost function defined as
$r(x, u) = q(x) + u^T R u$,
for a positive definite function
$q \colon \Omega \to \mathbb{R}_+$,
and a positive definite matrix $R \in \mathbb{S}_{++}$.

\begin{defi}[Admissible Control~\cite{beard_galerkin_1997-1}]
	Given the system in~\eqref{eq:system},
	a control $u \colon \Omega \to \mathbb{R}^m$
	is \textit{admissible}
	with respect to the cost function $r$
	in~\eqref{eq:performance_index},
	written $u \in \mathcal{A}_r(\Omega)$,
	if
	$u \in C^1(\Omega)$ stabilizes the system on $\Omega$,
	$u(0) = 0$,
	and
	$J(x_0; u) < \infty$ for all $x_0 \in \Omega$.
\end{defi}

For an admissible control $\mu \in \mathcal{A}_r(\Omega)$,
the value function can be defined as
\begin{equation}
	\label{eq:value_function}
	V(x; \mu) = J(x, \mu(\varphi(\cdot; x, \mu))).
\end{equation}


\bibliographystyle{IEEEtran}
\bibliography{note}

\end{document}

